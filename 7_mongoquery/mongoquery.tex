\documentclass[t]{beamer}

\usepackage[utf8]{inputenc}

\usepackage{color}
\definecolor{sh_comment}{rgb}{0.12, 0.38, 0.18 } %adjusted, in Eclipse: {0.25, 0.42, 0.30 } = #3F6A4D
\definecolor{sh_keyword}{rgb}{0.37, 0.08, 0.25}  % #5F1441
\definecolor{sh_string}{rgb}{0.06, 0.10, 0.98} % #101AF9
\definecolor{links}{HTML}{2A1BFF}

\usepackage[T1]{fontenc}
\usepackage[lighttt]{lmodern}

\usepackage{hyperref}
\usepackage{listings}
\usepackage{ccicons}

% "define" Scala
\lstdefinelanguage{Scala}{
  morekeywords={abstract,case,catch,class,def,%
    do,else,extends,false,final,finally,%
    for,if,implicit,import,match,mixin,%
    new,null,object,override,package,%
    private,protected,requires,return,sealed,%
    super,this,throw,trait,true,try,%
    type,val,var,while,with,yield},
  otherkeywords={=>,<-,<\%,<:,>:,\#,@},
  sensitive=true,
  morecomment=[l]{//},
  morecomment=[n]{/*}{*/},
  morestring=[b]",
  morestring=[b]',
  morestring=[b]"""
}

\lstset{
  language=Scala,
  rulesepcolor=\color{black},
  rulesepcolor=\color{black},
  showspaces=false,showtabs=false,tabsize=2,showstringspaces=false,
  numberstyle=\tiny,numbers=left,
  basicstyle=\ttfamily\tiny,
  stringstyle=\color{sh_string}, 
  keywordstyle=\color{sh_keyword}\bfseries,
  commentstyle=\color{sh_comment}, %\itshape
  captionpos=b,
  xleftmargin=0.7cm, xrightmargin=0.5cm,
  lineskip=-0.3em,
  escapebegin={\lstsmallmath}, escapeend={\lstsmallmathend}
}

\hypersetup{colorlinks,linkcolor=,urlcolor=links}

\def\linline{\lstinline[basicstyle=\ttfamily]}

\newcommand\strong\textbf

\setlength{\parskip}{\smallskipamount}

\author{Mikhail Limanskiy}

\institute{\url{www.limansky.me}}

\date{\include{date}}


\title{Compile-time string interpolation}

\begin{document}

\begin{frame}
    \titlepage
\end{frame}

\begin{frame}[fragile]
\frametitle{MongoDB}
MongoDB is a document oriented database.

It stores JSON documents in the collections.
\begin{lstlisting}
db.people.insert({ name: "John Doe", age: 42 })
db.people.insert({ 
    name: "William Smith",
    age: 28,
    phone: [ "1234567", "7654321" ]
    })
db.people.insert({
    name: "Alice White",
    age: 29,
    address: {
        country: "UK",
        city: "London"
    }
})
db.people.insert({ name : "Ivan Petrov", age : 28 })
\end{lstlisting}

And it also uses JSON as a query language:
\begin{lstlisting}
db.people.find({ name: "John Doe"})
db.people.find({ age: { $lt : 30 }})
db.people.find({ phone: { $not: { $size : 0 } } })
db.people.aggregate([{ $group : { _id : "$age", count : {$sum : 1} } },
                     { $sort : { count : -1 } },
                     { $limit : 5 }
  ])
\end{lstlisting}
\end{frame}

\begin{frame}[fragile]
\frametitle{Queries in Scala}
Casbah is a MongoDB interface for Scala.  It can build DBObjects:

\begin{lstlisting}
val name = "John Doe"

val a = people.findOne(MongoDBObject("name" -> name))

val b = people.find(MongoDBObject("age" -> MongoDBObject("$lt" -> 30)))

// Using Casbah DSL
val c = people.find("age" $lt 30)

val d = people.find(MongoDBObject("phone" -> MongoDBObject(
                    "$not" -> MongoDBObject("$size" -> 0))
        ))

val e = people.aggregate(List(
  MongoDBObject("$group" ->
    MongoDBObject("_id" -> "$age", "count" -> MongoDBObject("$sum" -> 1))),
  MongoDBObject("$sort" -> MongoDBObject("count" -> -1)),
  MongoDBObject("$limit" -> 5)))
\end{lstlisting}

ReactiveMongo is another Scala MongoDB driver built on top of Akka.  It provides
non-blocking reactive API:

\begin{lstlisting}
// Future[BSONDocument]
val a = people.find(BSONDocument("name" -> "John Doe")).one[BSONDocument]

// Future[List[Person]]
val b = people.find(BSONDocument("age" -> BSONDocument("$lt" -> 30)))
              .cursor[Person].collect[List]()
\end{lstlisting}
\end{frame}

\begin{frame}[fragile]
\frametitle{MongoQuery}
MongoQuery provides string interpolators for building MongoDB objects.  It makes
queryes more compact and allows to make some compile time checks.

\begin{lstlisting}
import com.github.limansky.mongoquery.casbah._

val name = "John Doe"

val a = people.findOne(mq"{ name : $name }")

val b = people.find(mq"{age : { $$lt : 30 } }")

val d = people.find(mq"{ phone : { $$not : { $$size : 0 }}}")

val e = people.aggregate(List(
  mq"""{ $$group : { _id : "$$age", count : { $$sum : 1 }}}""",
  mq"{ $$sort : { count : -1 }}",
  mq"{ $$limit : 5}"))
\end{lstlisting}

\end{frame}

\begin{frame}
\frametitle{Why to use string interpolation?}
Pros
\begin{itemize}
\item You want to use existing language
\end{itemize}
Cons
\begin{itemize}
\item You need to parse language yourself
\end{itemize}
\end{frame}

\begin{frame}[fragile]
\frametitle{String interpolators in Scala}
String interpolation converts into a corresponding method call on an instace of
StringContext class.

So, it is simple to define own interpolator:
\begin{lstlisting}
implicit class MongoHelper(val sc: StringContext) extends AnyVal {
  def mq(args: Any*): DBObject = parseQuery(sc.parts, args)
}
\end{lstlisting}
\end{frame}

\begin{frame}[fragile]
\frametitle{Puting it into macros}
\begin{lstlisting}
implicit class MongoHelper(val sc: StringContext) extends AnyVal {
  def a(args: Any*): DBObject = macro MongoHelper.mq_impl
}

object MongoHelper {
  def mq_impl(c: Context)(args: c.Expr[Any]*): c.Expr[DBObject] = ???
}
\end{lstlisting}
\end{frame}

\begin{frame}{The end}
\centering
\textbf{That's all folks!}

Code examples and this presentation are available at \url{http://github.com/limansky/trainings}\\~\\

\scriptsize{\ccbysa \ Mike Limanskiy, 2014-2015.}
\vfill
\tiny
This work is licensed under a \href{http://creativecommons.org/licenses/by-sa/4.0/}{Creative Commons Attribution-ShareAlike 4.0 International License}.

Created with \LaTeXe.
\end{frame}



\end{document}
