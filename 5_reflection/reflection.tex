\documentclass{beamer}
\usepackage[utf8]{inputenc}

\documentclass[t]{beamer}

\usepackage[utf8]{inputenc}

\usepackage{color}
\definecolor{sh_comment}{rgb}{0.12, 0.38, 0.18 } %adjusted, in Eclipse: {0.25, 0.42, 0.30 } = #3F6A4D
\definecolor{sh_keyword}{rgb}{0.37, 0.08, 0.25}  % #5F1441
\definecolor{sh_string}{rgb}{0.06, 0.10, 0.98} % #101AF9
\definecolor{links}{HTML}{2A1BFF}

\usepackage[T1]{fontenc}
\usepackage[lighttt]{lmodern}

\usepackage{hyperref}
\usepackage{listings}
\usepackage{ccicons}

% "define" Scala
\lstdefinelanguage{Scala}{
  morekeywords={abstract,case,catch,class,def,%
    do,else,extends,false,final,finally,%
    for,if,implicit,import,match,mixin,%
    new,null,object,override,package,%
    private,protected,requires,return,sealed,%
    super,this,throw,trait,true,try,%
    type,val,var,while,with,yield},
  otherkeywords={=>,<-,<\%,<:,>:,\#,@},
  sensitive=true,
  morecomment=[l]{//},
  morecomment=[n]{/*}{*/},
  morestring=[b]",
  morestring=[b]',
  morestring=[b]"""
}

\lstset{
  language=Scala,
  rulesepcolor=\color{black},
  rulesepcolor=\color{black},
  showspaces=false,showtabs=false,tabsize=2,showstringspaces=false,
  numberstyle=\tiny,numbers=left,
  basicstyle=\ttfamily\tiny,
  stringstyle=\color{sh_string}, 
  keywordstyle=\color{sh_keyword}\bfseries,
  commentstyle=\color{sh_comment}, %\itshape
  captionpos=b,
  xleftmargin=0.7cm, xrightmargin=0.5cm,
  lineskip=-0.3em,
  escapebegin={\lstsmallmath}, escapeend={\lstsmallmathend}
}

\hypersetup{colorlinks,linkcolor=,urlcolor=links}

\def\linline{\lstinline[basicstyle=\ttfamily]}

\newcommand\strong\textbf

\author{Mikhail Limanskiy}

\institute{SymphonyTeleca}

\date{\today}


\title{Scala reflection and macros}

\author{Mikhail Limanskiy}

\institute{SymphonyTeleca}

\date{\today}

\begin{document}

\begin{frame}
    \titlepage
\end{frame}

\section{Reflection}

\begin{frame}[fragile]
\frametitle{Java\texttrademark generics limitations}

In Java all generic type information is removed in compile time \cite{erasure}.  As result, there no type info for
parametrized types in JVM.  As result you cannot use type parameters for type checks or instantiation:

\begin{lstlisting}[language=Java]
public <E> void append(List<E> list, Class<E> cls) throws Exception {
//  E elem = new E(); // will not compile
    E elem = cls.newInstance();   // OK
    list.add(elem);
}

public <T extends Foo> T getByType(List<Foo> foos, Class<T> clazz) {
    List<T> result = new ArrayList<>();
    for (Foo foo : foos) {
//      if (foo instanceOf T) result.add(foo);  // will not compile
        if (clazz.instanceOf(foo)) result.add(foo);
    }
    return result;
}
\end{lstlisting}
\end{frame}

\begin{frame}[fragile]
\frametitle{Type erasure affects Scala}
This code will compile with warnings, but not work:
\begin{lstlisting}
def foo(item: Option[Any]) = item match {
  case t: Option[Int] => println("int")
  case t: Option[Boolean] => println("bool")
  case _ => println("other")
}
foo(Some(false))
\end{lstlisting}
\begin{lstlisting}[breaklines=true]
<console>:8: warning: non-variable type argument Int in type pattern Option[Int] is unchecked since it is eliminated by erasure
  case t: Option[Int] => println("int")
          ^
<console>:9: warning: non-variable type argument String in type pattern Option[String] is unchecked since it is eliminated by erasure
  case t: Option[String] => println("bool")
          ^
<console>:9: warning: unreachable code
  case t: Option[String] => println("bool")
                                   ^
\end{lstlisting}
\end{frame}

\begin{frame}[fragile]
\frametitle{Reflection universes}
Scala reflection API contains two universes: Runtime and Compile-time.

\begin{itemize}
\item Runtime reflection is used to get class information at program runtime. It's a same type of reflection as in Java.
\item Compile-time reflection allows you to get type information during compilation, and is used with Scala Macros.
\end{itemize}
\end{frame}

\begin{frame}[fragile]
\frametitle{Obtaining runtime type}
\begin{lstlisting}
def foo[T: TypeTag](item: Option[T]) = {
  val tt = typeOf[T]
  item match {
    case t if tt =:= typeOf[Boolean] => println("a bool")
    case t if tt =:= typeOf[Int]     => println("an int")
    case t if tt <:< typeOf[Seq[_]]  => println(s"a sequence ($tt)")
    case _                           => println("other")
  }
}

foo(Some(5)) // prints "a bool"
foo(Some(List(5,2))) // prints "a sequence (List[Int])"
\end{lstlisting}
\end{frame}

\begin{frame}{Bibliography}
\begin{thebibliography}{00}
\bibitem{scala}Martin Odersky, Lex Spoon, Bill Venners:
\emph{Programming in Scala},
Artima, 2nd edition, 2011
\bibitem{erasure}Java\texttrademark  Tutorials: \emph{Type erasure}, \url{http://docs.oracle.com/javase/tutorial/java/generics/erasure.html}
\bibitem{reflect}Heather Miller, Eugene Burmako, Philipp Haller: \emph{Scala reflection tutorial}, \url{http://docs.scala-lang.org/overviews/reflection/overview.html}
\end{thebibliography}
\end{frame}

\begin{frame}{The end}
\centering
\textbf{That's all folks!}

Code examples and this presentation are available at \url{http://github.com/limansky/trainings}\\~\\

\scriptsize{\ccbysa \ Mike Limanskiy, 2014-2015.}
\vfill
\tiny
This work is licensed under a \href{http://creativecommons.org/licenses/by-sa/4.0/}{Creative Commons Attribution-ShareAlike 4.0 International License}.

Created with \LaTeXe.
\end{frame}



\end{document}
