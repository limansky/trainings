\documentclass[t]{beamer}

\usepackage[utf8]{inputenc}

\usepackage{color}
\definecolor{sh_comment}{rgb}{0.12, 0.38, 0.18 } %adjusted, in Eclipse: {0.25, 0.42, 0.30 } = #3F6A4D
\definecolor{sh_keyword}{rgb}{0.37, 0.08, 0.25}  % #5F1441
\definecolor{sh_string}{rgb}{0.06, 0.10, 0.98} % #101AF9
\definecolor{links}{HTML}{2A1BFF}

\usepackage[T1]{fontenc}
\usepackage[lighttt]{lmodern}

\usepackage{hyperref}
\usepackage{listings}
\usepackage{ccicons}

% "define" Scala
\lstdefinelanguage{Scala}{
  morekeywords={abstract,case,catch,class,def,%
    do,else,extends,false,final,finally,%
    for,if,implicit,import,match,mixin,%
    new,null,object,override,package,%
    private,protected,requires,return,sealed,%
    super,this,throw,trait,true,try,%
    type,val,var,while,with,yield},
  otherkeywords={=>,<-,<\%,<:,>:,\#,@},
  sensitive=true,
  morecomment=[l]{//},
  morecomment=[n]{/*}{*/},
  morestring=[b]",
  morestring=[b]',
  morestring=[b]"""
}

\lstset{
  language=Scala,
  rulesepcolor=\color{black},
  rulesepcolor=\color{black},
  showspaces=false,showtabs=false,tabsize=2,showstringspaces=false,
  numberstyle=\tiny,numbers=left,
  basicstyle=\ttfamily\tiny,
  stringstyle=\color{sh_string}, 
  keywordstyle=\color{sh_keyword}\bfseries,
  commentstyle=\color{sh_comment}, %\itshape
  captionpos=b,
  xleftmargin=0.7cm, xrightmargin=0.5cm,
  lineskip=-0.3em,
  escapebegin={\lstsmallmath}, escapeend={\lstsmallmathend}
}

\hypersetup{colorlinks,linkcolor=,urlcolor=links}

\def\linline{\lstinline[basicstyle=\ttfamily]}

\newcommand\strong\textbf

\setlength{\parskip}{\smallskipamount}

\author{Mikhail Limanskiy}

\institute{\url{www.limansky.me}}

\date{\include{date}}


\title{Introduction to Akka}

\begin{document}

\begin{frame}
    \titlepage
\end{frame}

\Section{About}

\begin{frame}{About Akka}
\href{http://www.akka.io}{Akka} is an open-source toolkit and runtime simplifying the construction of concurrent
and distributed applications on the JVM. Akka supports multiple programming models
for concurrency, but it emphasizes actor-based concurrency, with inspiration drawn from Erlang.\\~\\

Akka is written in Scala, but provides bindings both for Scala and Java.\\~\\

First public release 0.5 was announced in January of 2010. Current release 2.3.9 (January 2015).
\end{frame}

\Section{Actors}

\begin{frame}{Actors Model}
The Actor model adopts the philosophy that everything is an \emph{actor}.  In Akka all \linline{Actor}s
live in \linline{ActorSystem}.\\~\\

Actor can:
\begin{itemize}
\item handle a message from it's mailbox
\item send message to another actors
\item create other actors
\item watch if the child actor is dead and recreate it
\end{itemize}
Actors interacts with each other \strong{only} by sending messages.\\~\\

Actors are executed concurrently.  It allows to run actors not only on different CPUs, but on the
different hosts in the network.
\end{frame}

\begin{frame}[fragile]
\frametitle{Props and ActorRef}
You shall never create actors using \linline{new}. You need use \linline{actorOf} method of the
\linline{ActorSystem} or the actor context.\\~\\

\linline{Props} class specify which actor will be created.\\~\\

As result you will receive \linline{ActorRef} instance which is a proxy object for the real actors.
It has methods to send messages:
\begin{itemize}
\item \linline{!} -- send message and forget (also known as \linline{tell}).
\item \linline{?} -- send message and get \linline{Future} for result (also known as \linline{ask}).
\item \linline{pipeTo} -- send future result.
\end{itemize}
\begin{example}
\begin{lstlisting}
val system = ActorSystem("testsystem")
val hello = system.actorOf(Props[Hello])
hello ! "World"
\end{lstlisting}
\end{example}
\end{frame}

\begin{frame}[fragile]
\frametitle{Actor class}
The \linline{Actor} class is abstract base for all actors. Main methods are:
\begin{itemize}
  \item \linline{receive: Receive} -- abstract method handling incoming messages one by one
    from actor mailbox
    (where \linline{type Receive = PartialFunction[Any, Unit]}).
  \item \linline{self: ActorRef} -- returns self reference
  \item \linline{sender: ActorRef} -- reference to actor of the last received message.
  \item \linline{context: ActorContext} -- provides actor context.
\end{itemize}
\begin{lstlisting}
class Hello extends Actor {
  var count = 0

  def receive = {
    name: String => println ("Hello, " + name)
    amount: Int =>
      count += amount
      sender ! count
  }
}
\end{lstlisting}
\end{frame}

\begin{frame}[fragile]
\frametitle{Behaviors}
You can change actor behavior on runtime by switching to another message handling function instead
of receive. Use following \linline{context} methods:
\begin{itemize}
\item \linline{become(behavior:Receive, discardOld:Boolean=true)}
\item \linline{unbecome(): Unit}
\end{itemize}

It's quite useful to save state in partially applied parameters of function returns \linline{Receive}.
\begin{example}
\begin{lstlisting}
def receive = {
  case DieAfterFiveEvents =>
    context become counter(5)
}

def counter(n: Integer): Receive = {
  case e =>
    if (n == 0) context stop self
    else context become counter(n - 1)
    // do something else
}
\end{lstlisting}
\end{example}
\end{frame}

\begin{frame}[fragile]
\frametitle{Code example}
\begin{lstlisting}[name=person]
object Person {
  case class Start(left: List[ActorRef])
  case class Introduce(name: String)
  case class Bye(who: ActorRef)
  case object SaySomething
  case object AskTime
}

class Person(name: String) extends Actor {

  import Person._
  import context.dispatcher
  import akka.pattern.ask

  implicit val timeout = Timeout(100.millis)

  def receive = {
    case Start(lst) =>
      val d = Random.nextInt(5)
      context.system.scheduler.scheduleOnce(d.seconds, self, SaySomething)
      context become speaking(3, lst.toSet - self)
  }
\end{lstlisting}
\end{frame}

\begin{frame}[fragile]
\frametitle{Code example, part 2}
\begin{lstlisting}[name=person]
def speaking(c: Int, persons: Set[ActorRef]): Receive = {
  case Introduce(n) => sender ! s"Nice to meet you, $n! I'm $name"

  case AskTime =>
    val time = new Date
    sender ! s"I'm $name, it's $time"

  case Bye(a) =>
    val left = persons - a
    if (left.isEmpty) context.system.shutdown
    else context become speaking(c, left)

  case SaySomething =>
    if (c > 0) {
      val msg = if (Random.nextBoolean) {
        println(s"Hello! My name is $name")
        Introduce(name)
      } else {
        println(s"I'm $name. What time is it?")
        AskTime
      }
      val a = persons.toList(Random.nextInt(persons.size))
      val f = a ? msg
      f.onSuccess { case reply =>
        println(reply)
        val d = Random.nextInt(5)
        context.system.scheduler.scheduleOnce(d.seconds, self, SaySomething)
        context become speaking(c - 1, persons)
      }
    } else {
      println("Bye-bye from " +  name)
      persons foreach (_ ! Bye(self))
      context.stop(self)
    }
}
\end{lstlisting}
\end{frame}

\begin{frame}[fragile]
\frametitle{Starting example actors}
\begin{lstlisting}
import akka.actor._

object Main extends App {
  val names = List("John", "Paul", "George", "Ringo")

  val system = ActorSystem("Beatles")

  val actors = names.map(name => system.actorOf(Props(classOf[Person], name)))

  actors foreach (_ ! Person.Start(actors))
}
\end{lstlisting}
\end{frame}

\begin{frame}[fragile]
\frametitle{Testing actors}
  Akka testkit module allows you to test actors using testing different test frameworks
such as Scalatest, Specs2 or even JUnit.
\begin{lstlisting}
import akka.actor.{ ActorSystem, Actor, Props }
import akka.testkit.{ TestKit, TestProbe, ImplicitSender }
import org.scalatest.{ FlatSpecLike, BeforeAndAfterAll }
import scala.concurrent.duration._

class PersonTest(aSystem: ActorSystem) extends TestKit(aSystem) with ImplicitSender
          with FlatSpecLike with BeforeAndAfterAll {

  def this() = this(ActorSystem("PersonTestSystem"))
  override def afterAll = system.shutdown

  "Person" should "not response before start" in {
    system.actorOf(Props(classOf[Person], "Test")) ! Person.Introduce("Petr")
    expectNoMsg(500.millis)
  }
  it should "say something after start" in {
    val p = system.actorOf(Props(classOf[Person], "Test"))
    val probe = TestProbe()
    p ! Person.Start(List(p, probe.ref))
    probe.expectMsgAnyOf(5.seconds, Person.AskTime, Person.Introduce("Test"))
    probe.send(p, "Ok!")
  }
  it should "response on introduction" in {
    val p = system.actorOf(Props(classOf[Person], "Test"))
    p ! Person.Start(List(p, TestProbe().ref))
    p ! Person.Introduce("Petr")
    expectMsg("Nice to meet you, Petr! I'm Test.")
  }
}
\end{lstlisting}
\end{frame}

\section{Resilience}

\begin{frame}[fragile]
\frametitle{Death watch}
Sometimes actors can fail, or stop. It is possible to notify one actor about another actor
termination using \linline{context.watch} / \linline{context.unwatch} methods.
\begin{lstlisting}
class Foo(barProps: Props) extends Actor {
  val bar = context.actorOf(barProps)
  context.watch(bar)

  def receive = {
    ...
    case Terminated(a) => println("received terminated")
  }
}
\end{lstlisting}
\end{frame}

\begin{frame}[fragile]
\frametitle{Supervision}
Each actor supervise it's children.  In case of exception in child it can perform some action,
depends on current supervisor strategy:
\begin{enumerate}
\item Resume. Subordinate continue running keeping it's state.
\item Restart. Subordinate is restarted clearing out it's state.
\item Stop. Subordinate is stopped permanently.
\item Escalate. Stops supervisor itself, pass error to parent.
\end{enumerate}
You can override supervision strategy in your actor:
\begin{lstlisting}
override val supervisorStrategy =
  OneForOneStrategy(maxNrOfRetries = 10, withinTimeRange = 1 minute) {
    case _: ArithmeticException => Resume
    case _: NullPointerException => Restart
    case _: IllegalArgumentException => Stop
    case _: Exception => Escalate
}
\end{lstlisting}
\end{frame}

\begin{frame}[fragile]
\frametitle{Routing}
Sometimes you need one master actor and several workers.  Routers allows you to have one
entry point for workers pool.
\begin{lstlisting}
val router: ActorRef = context.actorOf(RoundRobinPool(3).props(Props[Worker]))
router ! "Handle me!"
\end{lstlisting}
There are several predefined routing strategies, but you can implement your own one.
\begin{itemize}
\item Round-robin
\item Random
\item Smallest mailbox
\item Broadcast
\item Scatter-Gather (send to all, take first)
\item Consistent hashing
\end{itemize}
\end{frame}

\begin{frame}[fragile]
\frametitle{Agents. Sharing state}
Akka agents is a shared mutable state holders.  Agents provide synchronous reading and asynchronous updates.
\begin{lstlisting}[name=agents]
val a = Agent(4)
println(a())

a send 34
a send (_ + 8)
\end{lstlisting}
It's also possible to get \linline{Future} with updated value using \linline{alter}:
\begin{lstlisting}[name=agents]
val f: Future[Int] = a alter (_ * 5)
\end{lstlisting}
In addition you can use monadic syntax:
\begin{lstlisting}[name=agents]
val b = a map (_ - 10)

val c = for {
    v1 <- a
    v2 <- b
} yield v1 + v2
\end{lstlisting}
\end{frame}

\begin{frame}[fragile]
\frametitle{Network IO}
Akka IO package was developed in collaboration between Akka and \href{http://spray.io}{spray.io} teams. 
It's completely actor based. Every IO driver has a special actor called \emph{manager}:
\linline{val manager = IO(Tcp)}.

\begin{example}
\begin{lstlisting}
import java.net.InetSocketAddress
import akka.io.{IO, Tcp}
import akka.util.ByteString

class Client(addr: InetSocketAddress) extends Actor {
  import Tcp._
  import context.system

  IO(Tcp) ! Connect(addr)

  def receive = {
    case CommandFailed(_: Connect) => context stop self

    case Connected(remote, local) =>
      val conn = sender()
      conn ! Register(self)
      context become {
        case Received(data) => println("Got: " + data.utf8String)

        case data: ByteString => conn ! Write(data)

        case "close" => conn ! Close
      }
  }
}
\end{lstlisting}
\end{example}
\end{frame}

\begin{frame}[fragile]
\frametitle{TCP server example}
\begin{lstlisting}
class Server extends Actor {
  import Tcp._
  import context.system

  IO(Tcp) ! Bind(self, new InetSocketAddress("localhost", 12345))

  def receive = {
    case Connected(_, _) =>
      val handler = context.actorOf(Props[HelloHandler])
      sender ! Register(handler)
  }
}

class HelloHandler extends Actor {
  import Tcp._

  def receive = {
    case Received(_) => sender ! Write(ByteString("Hello!"))
    case PeerClosed => context stop self
  }
}
\end{lstlisting}
\end{frame}

\begin{frame}[fragile]
\frametitle{Actor path and selection}
Since actors in Akka has hierarchy it is possible to create unique path for each actor.
For example: \linline{akka://MySystem/user/actorA/actorB}. You can also use relative paths
and paths for current context: \linline{../actorC} or \linline{/user/actorA/actorC}.\\~\\

Having actor path you can send message to the actor:
\begin{lstlisting}
context.actorSelection("/user/myWorker") ! "hello"

context.actorSelection("../*") ! NotifyAll
\end{lstlisting}
\end{frame}

\begin{frame}[fragile]
\frametitle{Remoting}
You can also select actors from remote actor system:
\begin{lstlisting}
val remote = context.actorSelection(
  "akka.tcp://OtherSystem@192.168.1.2:2552/user/service")

remote ! "Hello remote"
\end{lstlisting}
To make this work you need to set actor provider to \linline{akka.remote.RemoteActorRefProvider}
in your application.conf file:
\begin{lstlisting}
akka {
  actor {
    provider = "akka.remote.RemoteActorRefProvider"
  }
  remote {
    enabled-transports = ["akka.remote.netty.tcp"]
    netty.tcp {
      hostname = "192.168.1.1"
      port = 2552
    }
  }
}
\end{lstlisting}
\end{frame}

\begin{frame}{The end}
\centering
\textbf{That's all folks!}

Code examples and this presentation are available at \url{http://github.com/limansky/trainings}\\~\\

\scriptsize{\ccbysa \ Mike Limanskiy, 2014-2015.}
\vfill
\tiny
This work is licensed under a \href{http://creativecommons.org/licenses/by-sa/4.0/}{Creative Commons Attribution-ShareAlike 4.0 International License}.

Created with \LaTeXe.
\end{frame}



\end{document}
