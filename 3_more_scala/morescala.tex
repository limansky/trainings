\documentclass{beamer}
\usepackage[utf8]{inputenc}

\documentclass[t]{beamer}

\usepackage[utf8]{inputenc}

\usepackage{color}
\definecolor{sh_comment}{rgb}{0.12, 0.38, 0.18 } %adjusted, in Eclipse: {0.25, 0.42, 0.30 } = #3F6A4D
\definecolor{sh_keyword}{rgb}{0.37, 0.08, 0.25}  % #5F1441
\definecolor{sh_string}{rgb}{0.06, 0.10, 0.98} % #101AF9
\definecolor{links}{HTML}{2A1BFF}

\usepackage[T1]{fontenc}
\usepackage[lighttt]{lmodern}

\usepackage{hyperref}
\usepackage{listings}
\usepackage{ccicons}

% "define" Scala
\lstdefinelanguage{Scala}{
  morekeywords={abstract,case,catch,class,def,%
    do,else,extends,false,final,finally,%
    for,if,implicit,import,match,mixin,%
    new,null,object,override,package,%
    private,protected,requires,return,sealed,%
    super,this,throw,trait,true,try,%
    type,val,var,while,with,yield},
  otherkeywords={=>,<-,<\%,<:,>:,\#,@},
  sensitive=true,
  morecomment=[l]{//},
  morecomment=[n]{/*}{*/},
  morestring=[b]",
  morestring=[b]',
  morestring=[b]"""
}

\lstset{
  language=Scala,
  rulesepcolor=\color{black},
  rulesepcolor=\color{black},
  showspaces=false,showtabs=false,tabsize=2,showstringspaces=false,
  numberstyle=\tiny,numbers=left,
  basicstyle=\ttfamily\tiny,
  stringstyle=\color{sh_string}, 
  keywordstyle=\color{sh_keyword}\bfseries,
  commentstyle=\color{sh_comment}, %\itshape
  captionpos=b,
  xleftmargin=0.7cm, xrightmargin=0.5cm,
  lineskip=-0.3em,
  escapebegin={\lstsmallmath}, escapeend={\lstsmallmathend}
}

\hypersetup{colorlinks,linkcolor=,urlcolor=links}

\def\linline{\lstinline[basicstyle=\ttfamily]}

\newcommand\strong\textbf

\setlength{\parskip}{\smallskipamount}

\author{Mikhail Limanskiy}

\institute{\url{www.limansky.me}}

\date{\include{date}}


\title{Advanced Scala programming}
\subtitle{extractors, reflection, macros, parsers}

\author{Mikhail Limanskiy}

\institute{SymphonyTeleca}

\date{\today}

\begin{document}

\begin{frame}
    \titlepage
\end{frame}

\section{Extractors}

\begin{frame}[fragile]
\frametitle{Recap: pattern matching}
Pattern matching is a Scala feature similar to switch/case operator, but allows you to match
using much more complex rules:
\begin{example}
\begin{lstlisting}
val l = List(1, 8, 4,  2, 5, 12)
l.sorted find (_ > 3) match {
  case Some(x) => println("$x is a smallest value greater than 3 in the list")
  case None    => println("There are no values greater than 3")
}

def getSecond(l: List[A]): Option[A] = l match {
  case _::x::_ => Some(x)
  case _       => None
}
\end{lstlisting}
\end{example}
\end{frame}

\begin{frame}[fragile]
\frametitle{Case classes}
You can match on any classes, but using \linline{case class}es allows you to match on constructors.
Also, when you define case class:
\begin{itemize}
\item creates factory method with a same name (you can create instances without using new)
\item creates values for each constructor parameter
\end{itemize}
\begin{example}
\begin{lstlisting}
sealed abstract class Result[T]
case class Success[T](value: T) extends Result
case class Error(message: String) extends Result

def divide(x: Int, y: Int) = 
  if (y == 0) Error("Division by 0") else Success(x / y)

divide(25, 7) match {
  case Success(r) => println("Result is $r")
  case Failure(m) => println(m)
}
\end{lstlisting}
\end{example}
\end{frame}

\begin{frame}[fragile]
\frametitle{Extractors}

Extractors allows you to use pattern matching without case classes. Extractor is an object with
method \linline{unapply}.

\begin{example}
\begin{lstlisting}
import scala.util.Try

object Twice {
  def apply(x: Int) = x * 2
  def unapply(z: Int) = if (z % 2 == 0) Some(z / 2) else None
}

object AsInt {
  def unapply(s: String) = Try(s.toInt).toOption
}

def printHalf(x: Int) = x match {
  case Twice(y) => println(s"$x is two times $y")
  case _        => println("x is odd")
}

def foo(s: String) = s match {
  case AsInt(Twice(y)) => println("The half is " + y)
  case _               => println("Wrong input")
}
\end{lstlisting}
\end{example}
\end{frame}

\section{Parsers}

\begin{frame}[fragile]
\frametitle{Combinatoric parsers}
Combinatoric parsers is the easy way to build complex parsers from the small parsers blocks.
Main combinators are:
\begin{itemize}
\item \linline{~} -- sequential composition
\item \linline{|} -- alternative composition
\item \linline{~>} and \linline{<~} -- sequence ignoring right or left value.
\item \linline{rep} or \linline{*} -- repeat parser
\item \linline{rep1} or \linline{+} -- repeat at least once
\item \linline{opt} or \linline{?} -- optional parser
\item \linline{repsep} -- repeated parser with separator
\item \linline{^^} -- convert parsed value using function.
\end{itemize}
\end{frame}

\begin{frame}[fragile]
\frametitle{Parsing CSV example}
\begin{lstlisting}
import scala.util.parsing.combinator.RegexParsers
import scala.io.Source

abstract class Cell
case class NumCell(number: Double) extends Cell
case class StringCell(string: String) extends Cell

trait CSVParser extends RegexParsers {
  override def skipWhitespace = false
  def lf = "\n"
  def space = "[ \t]*".r

  def number: Parser[NumCell] = 
    space ~> """-?\d+(\.\d*)?""".r <~ space ^^ { x => NumCell(x.toDouble) }
  
  def string: Parser[StringCell] = """[^,\n]*""".r ^^ { StringCell }

  def cell: Parser[Cell] = number | string
  def line: Parser[List[Cell]] = repsep(cell, ",") <~ lf
  def file: Parser[List[List[Cell]]] = rep(line)
}

object FileParser extends App with CSVParser {

  val str = Source.fromFile(args.head).mkString
  parseAll(file, str) match {
    case Success(result, _) => result foreach (l => println(l.mkString("\t")))
    case fail: NoSuccess    => println("Parsing failed" + fail.msg)
  }
}
\end{lstlisting}
\end{frame}

\begin{frame}[fragile]
\frametitle{Calculator using parsers}
Defining a model:
\begin{lstlisting}[name=calc]
import scala.util.parsing.combinator.JavaTokenParsers

abstract class Expr {
  val eval: Double
}

case class Add(a: Expr, b: Expr) extends Expr {
  override val eval = a.eval + b.eval
}

case class Sub(a: Expr, b: Expr) extends Expr {
  override val eval = a.eval - b.eval
}

case class Mul(a: Expr, b: Expr) extends Expr {
  override val eval = a.eval * b.eval
}

case class Div(a: Expr, b: Expr) extends Expr {
  override val eval = a.eval / b.eval
}

case class Num(a: Double) extends Expr {
  override val eval = a
}
\end{lstlisting}
\end{frame}

\begin{frame}[fragile]
\frametitle{Calculator using parsers, continuation}
Defining parser:
\begin{lstlisting}[name=calc]
trait CalcParser extends JavaTokenParsers {
  def num = floatingPointNumber ^^ (s => Num(s.toDouble))

  def plus: Parser[Expr => Expr] = "+" ~> term ^^ (x => Add(_, x))

  def minus: Parser[Expr => Expr] = "-" ~> term ^^ (x => Sub(_, x))

  def times: Parser[Expr => Expr] = "*" ~> factor ^^ (x => Mul(_, x))

  def div: Parser[Expr => Expr] = "/" ~> factor ^^ (x => Div(_, x))

  def sum: Parser[Expr] = term ~ rep(plus | minus) ^^ {
      case a ~ b => b.foldLeft(a)((i, f) => f(i))
  }

  def term: Parser[Expr] = factor ~ rep(times | div) ^^ {
      case a ~ b => b.foldLeft(a)((i, f) => f(i))
  }

  def factor = num | "(" ~> sum <~ ")"
}

object Calculator extends App with CalcParser {
  parseAll(sum, args.mkString) match {
    case Success(e, _) => println("Result: " + e.eval)
    case fail: NoSuccess    => println("Parsing failed: " + fail.msg)
  }
}
\end{lstlisting}
\end{frame}

\section{Reflection}

\begin{frame}[fragile]
\frametitle{Java\texttrademark generics limitations}

In Java all generic type information is removed in compile time \cite{erasure}.  As result, there no type info for
parametrized types in JVM.  As result you cannot use type parameters for type checks or instantiation:

\begin{lstlisting}[language=Java]
public <E> void append(List<E> list, Class<E> cls) throws Exception {
//  E elem = new E(); // will not compile
    E elem = cls.newInstance();   // OK
    list.add(elem);
}

public <T extends Foo> T getByType(List<Foo> foos, Class<T> clazz) {
    List<T> result = new ArrayList<>();
    for (Foo foo : foos) {
//      if (foo instanceOf T) result.add(foo);  // will not compile
        if (clazz.instanceOf(foo)) result.add(foo);
    }
    return result;
}
\end{lstlisting}
\end{frame}

\begin{frame}[fragile]
\frametitle{Type erasure affects Scala}
This code will compile with warning, but not work:
\begin{lstlisting}
class Base()
class Child1() extends Base
class Child2() extends Base

def foo(item: Option[Base]) = item match {
  case Some(Child1) => println("child1")
  case Some(Child2) => println("child2")
  case None         => println("none")
}
\end{lstlisting}
\end{frame}

\begin{frame}{Bibliography}
\begin{thebibliography}{00}
\bibitem{scala}Martin Odersky, Lex Spoon, Bill Venners:
\emph{Programming in Scala},
Artima, 2nd edition, 2011
\bibitem{erasure}Java\texttrademark  Tutorials: \emph{Type erasure}, \url{http://docs.oracle.com/javase/tutorial/java/generics/erasure.html}
\end{thebibliography}
\end{frame}

\begin{frame}{The end}
\centering
\textbf{That's all folks!}

Code examples and this presentation are available at \url{http://github.com/limansky/trainings}\\~\\

\scriptsize{\ccbysa \ Mike Limanskiy, 2014-2015.}
\vfill
\tiny
This work is licensed under a \href{http://creativecommons.org/licenses/by-sa/4.0/}{Creative Commons Attribution-ShareAlike 4.0 International License}.

Created with \LaTeXe.
\end{frame}



\end{document}
