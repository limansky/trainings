\documentclass[t]{beamer}

\usepackage[utf8]{inputenc}

\usepackage{color}
\definecolor{sh_comment}{rgb}{0.12, 0.38, 0.18 } %adjusted, in Eclipse: {0.25, 0.42, 0.30 } = #3F6A4D
\definecolor{sh_keyword}{rgb}{0.37, 0.08, 0.25}  % #5F1441
\definecolor{sh_string}{rgb}{0.06, 0.10, 0.98} % #101AF9
\definecolor{links}{HTML}{2A1BFF}

\usepackage[T1]{fontenc}
\usepackage[lighttt]{lmodern}

\usepackage{hyperref}
\usepackage{listings}
\usepackage{ccicons}

% "define" Scala
\lstdefinelanguage{Scala}{
  morekeywords={abstract,case,catch,class,def,%
    do,else,extends,false,final,finally,%
    for,if,implicit,import,match,mixin,%
    new,null,object,override,package,%
    private,protected,requires,return,sealed,%
    super,this,throw,trait,true,try,%
    type,val,var,while,with,yield},
  otherkeywords={=>,<-,<\%,<:,>:,\#,@},
  sensitive=true,
  morecomment=[l]{//},
  morecomment=[n]{/*}{*/},
  morestring=[b]",
  morestring=[b]',
  morestring=[b]"""
}

\lstset{
  language=Scala,
  rulesepcolor=\color{black},
  rulesepcolor=\color{black},
  showspaces=false,showtabs=false,tabsize=2,showstringspaces=false,
  numberstyle=\tiny,numbers=left,
  basicstyle=\ttfamily\tiny,
  stringstyle=\color{sh_string}, 
  keywordstyle=\color{sh_keyword}\bfseries,
  commentstyle=\color{sh_comment}, %\itshape
  captionpos=b,
  xleftmargin=0.7cm, xrightmargin=0.5cm,
  lineskip=-0.3em,
  escapebegin={\lstsmallmath}, escapeend={\lstsmallmathend}
}

\hypersetup{colorlinks,linkcolor=,urlcolor=links}

\def\linline{\lstinline[basicstyle=\ttfamily]}

\newcommand\strong\textbf

\setlength{\parskip}{\smallskipamount}

\author{Mikhail Limanskiy}

\institute{\url{www.limansky.me}}

\date{\include{date}}


\usepackage{tikz}
\usepackage{tikz-uml}
\usetikzlibrary{positioning}

\usetheme{metropolis}
%\usetheme{ofd}

\title{Introduction to Scala}

\begin{document}

\begin{frame}
    \titlepage
\end{frame}

\section{Why Scala}

\begin{frame}{About}

\begin{itemize}[<+->]
  \item Scala = SCAlable + LAnguage

  \item Scala is an object-functional programming and scripting language for general software applications, statically typed, designed to concisely express solutions in an elegant, type-safe and lightweight (low ceremonial) manner.

  \item Scala compiles into Java byte code and runs on JVM. It compatible with existing Java libraries (but of course you should prefer use native Scala libraries).

  \item Scala also can be compiled into JavaScript and run in browser or Node.js platform.

  \item And it also possible to compile Scala to native code.
\end{itemize}

\end{frame}

\begin{frame}{History}

\begin{itemize}[<+->]
  \item 2001 - The design of Scala started at the École Polytechnique Fédérale de Lausanne (EPFL) by Martin Odersky.
  \item 2003 - First release.
  \item 2006 - Scala 2.0 release.
  \item 2010 - Scala 2.8 release. New collection framework.
  \item 2013 - Scala 2.10 release. Futures and promises. Akka actors now main actors implementation.
  \item 2014 - Scala 2.11 release. Reflection, macros and quasiquotes.
  \item 2016 - Scala 2.12 release. Better Java 8 support, SAM types.
  \item 2018 - Scala 2.13 release (M4 now). New collections.
  \item 2019 - Scala 2.14 release.
  \item 2019 - Scala 3 aka Dotty.
\end{itemize}
\end{frame}

\section{Syntax}

\begin{frame}[fragile]
\frametitle{Hello world!}

\begin{lstlisting}
import scala.util.Random

object Hello extends App {
  println("Hello, world!")

  if (Random.nextDouble > 0.5) println("I'm so happy")

  for (i <- 1 to 5;
       j <- 1 to i) println(i + " " + j)

  Random.nextInt(5) match {
    case 1 => println("One")
    case 2 => println("Two")
    case _ => println("Somthing else")
  }

}
\end{lstlisting}

\end{frame}

\begin{frame}[fragile]
\frametitle{Variables and their types}
\begin{center}
\begin{tabular}[t]{l | c | c | c}
Name & Definition & Reassignment & Computation \\
\hline
Value & \texttt{val} & No & Immediatelly \\ \pause
Variable & \texttt{var} & Yes & Immediatelly \\ \pause
Lazy value & \texttt{lazy val} & No & First using \\ \pause
Definition & \texttt{def} & No & Each time
\end{tabular}
\end{center}
\pause
\begin{lstlisting}
// Long
val a = System.currentTimeMillis
// a = 10: compile error
|\pause|
lazy var b = System.currentTimeMillis
b = 10
|\pause|
def c = System.currentTimeMillis
|\pause|
lazy val d = System.currentTimeMillis
println(s"a=$a b=$b c=$c d=$d")
Thread.sleep(1000)
println(s"a=$a b=$b c=$c d=$d")
\end{lstlisting}

\end{frame}

\begin{frame}[fragile]
\frametitle{Classes, objects, and traits}

Scala has no static class members. But there are singleton objects. If object has a same name with a class is called companion object.

\begin{lstlisting}
class Foo(val x: Int, y: String) {
  val r = y.reverse
}

object Foo {
  def fromString(s: String) = new Foo(1, s)
|\onslide<2->|
  def apply(n: Int) = new Foo(n, "") |\onslide<1->|
}

val a = new Foo(5, "test")
val b = Foo.fromString("buzz")

// prints "tsetzzub"
println(a.r + b.r) |\onslide<2->|

val c = Foo(42)
\end{lstlisting}

\end{frame}

\begin{frame}[fragile]
\frametitle{Classes, objects, and traits}

There traits in Scala instead of Java interfaces. Traits can contain fields and abstract and concrete methods. Classes can be inherited from on class and several traits.

\begin{lstlisting}
abstract class A(a: Int) {
  def callMe(a: String): Unit
}
|\pause|
trait B {
  def foo: Int
}
|\pause|
trait C {
  def bar: String = "Let's go to the bar"
}
|\pause|
class D extends A with B with C
\end{lstlisting}

\end{frame}

\begin{frame}[fragile]
\frametitle{Multiple inheritance in Scala. Linearization}

\begin{columns}
  \begin{column}{0.48\textwidth}

\begin{tikzpicture}

\umlclass[type=abstract]{Base}{
  def foo: String
}{}
\umlclass[below left=0.8cm and -0.8cm of Base]{A}{
  def foo = "A"
}{}
\umlinherit[anchor1=90, anchor2=-125]{A}{Base}
\umlclass[below right=0.8cm and -0.8cm of Base]{B}{
  def foo = "B"
}{}
\umlinherit[anchor1=90, anchor2=-50]{B}{Base}
\umlclass[below=3cm of Base]{Foo}{}{}
\umlinherit[anchor2=-90]{Foo}{A}
\umlinherit[anchor2=-90]{Foo}{B}

\end{tikzpicture}
\pause

  \end{column}
  \begin{column}{0.48\textwidth}

\begin{lstlisting}
trait Base {
 def foo: String
}

trait A extends Base {
  override def foo: String = "A"
}

trait B extends Base {
  override def foo: String = "B"
}

class Foo extends A with B // foo == "B"
|\pause|
class Bar extends B with A // foo == "A"
|\pause|
class Baz extends A with B {
  override def foo = super[A].foo + super[B].foo
}
\end{lstlisting}

  \end{column}
\end{columns}

\end{frame}

\begin{frame}[fragile]
\frametitle{Parameterized types}

Types can have parameters.

\begin{lstlisting}
class Foo

// Can hold any class
class Holder[A](val a: A)

// Holder[Int]
val intHolder = new Holder(5)

// Can hold only Foo and it's children
class FooHolder[A <: Foo](val a: A)

// Can hold only Foo and it's parents
class FooParentHolder[A >: Foo](val a: A)
\end{lstlisting}

\pause
Functions can be parameterized as well

\begin{lstlisting}
object Foo {
  def applyFunction[A, B](v: A, f: A => B) = f(v)
}
\end{lstlisting}

\end{frame}

\begin{frame}[fragile]
\frametitle{Parameter variance}
Type parameters can have variance modifiers to specify inheriting behavior.
\begin{itemize}
\item \linline{class Test1[A](val a: A)} -- nonvariant subtyping (default)
\item \linline{class Test2[+A](val a: A)} -- covariant subtyping
\item \linline{class Test3[-A](val a: A)} -- contravariant subtyping
\end{itemize}
\begin{tikzpicture}
\umlsimpleclass{B}
\umlsimpleclass[y=-1.5]{C}
\umlsimpleclass[x=2.5]{Test1[B]}
\umlsimpleclass[x=2.5, y=-1.5]{Test1[C]}
\umlsimpleclass[x=5]{Test2[B]}
\umlsimpleclass[x=5, y=-1.5]{Test2[C]}
\umlsimpleclass[x=7.5]{Test3[B]}
\umlsimpleclass[x=7.5, y=-1.5]{Test3[C]}
\umlinherit{C}{B}
\umlinherit{Test2[C]}{Test2[B]}
\umlinherit{Test3[B]}{Test3[C]}
\end{tikzpicture}

\end{frame}

\subsection{OOP features}

\begin{frame}[fragile]
\frametitle{OOP example}

\begin{lstlisting}
abstract class Shape {
  def area(): Double
}

trait Colored {
  val color: String
}

class Circle(val r: Double) extends Shape {
  override def area() = math.Pi * r * r
}

class ColorCircle(val r: Double, val color: String) 
  extends Circle(r) 
  with Colored

object Shapes extends App {
  val circles = List(new Cicrle(10), new ColorCircle(3, "Red"))

  for (c <- circles) {
    // For Scala < 2.10
    // println("Circle with radius " + c.r + "has area: " + c.area)
    // For Scala 2.10, using string interpolation:
    println(s"Circle with radius ${c.r} has area: ${c.area}")
  }
}

\end{lstlisting}
\end{frame}

\begin{frame}{Class hierarchy}

\begin{tikzpicture}
\umlsimpleclass{Any}
\umlsimpleclass[x=-3, y=-1]{AnyVal}
\umlsimpleclass[x=3, y=-1]{AnyRef}
\umlsimpleclass[x=-4, y=-2.5]{Int}
\umlsimpleclass[x=-2, y=-2.5]{Double}
\umlsimpleclass[x=2, y=-2.5]{String}
\umlsimpleclass[x=4, y=-2.5]{List}
\umlsimpleclass[x=3, y=-4]{Null}
\umlsimpleclass[x=-1,y=-5]{Nothing}
\umlinherit{AnyVal}{Any}
\umlinherit{AnyRef}{Any}
\umlinherit{Int}{AnyVal}
\umlinherit{Double}{AnyVal}
\umlinherit{String}{AnyRef}
\umlinherit{List}{AnyRef}
\umlinherit{Null}{String}
\umlinherit{Null}{List}
\umlinherit{Nothing}{Double}
\umlinherit{Nothing}{Int}
\umlinherit{Nothing}{Null}
\end{tikzpicture}

\end{frame}


\begin{frame}[fragile]
\frametitle{Implicit conversions}

Scala values can be implicitly converted to another type if there is some converter available in current scope. E.g.
\linline{val range = 1 to 5}. It's the same with \linline{val range = 1.to(5)}. But type Int has no to method.
The reason is that \texttt{Int} value converted implicitly to \texttt{RichInt}.

\begin{example}
\begin{lstlisting}
implicit class StrMix(val s: String) extends AnyVal {
  def <||>(o: String): String =
    s.zip(o).map(t => t._1.toString + t._2.toString).mkString
}

// Usage. Result will be "hweolrllod"
println("hello" <||> "world")
\end{lstlisting}
\end{example}

\end{frame}

\subsection{Functional features}

\begin{frame}[fragile]
\frametitle{Scala is a functional language}
  Scala functions are first class members:
  \begin{itemize}
    \item You can create variables of function types.
    \item You can pass functions as a parameters.
    \item You can return functions as a result of other function.
    \item Anonimous functions ($\lambda$-functions).
  \end{itemize}
  \begin{example}
\begin{lstlisting}
class Updatable[T](v: T) {
  def updated(f: T => T) = new Updatable(f(v))
}

val a = new Updatable(10)
val b = a.updated(_ + 5) // Updatable(15)
\end{lstlisting}
  \end{example}
\end{frame}

\begin{frame}[fragile]
\frametitle{More complex example}

    \begin{lstlisting}
def swapParams[A,B,C](f: (A,B) => C): (B,A) => C = 
  (b, a) => f(a, b)

def repeat(a: String, times: Int) = 
  (for (i <- 1 to times) yield a) mkString " "

val repeatS = swapParams(repeat)
    \end{lstlisting}

\end{frame}

\begin{frame}[fragile]
\frametitle{Under the hood}

When you assign function to variable, it's type is a one of the traits \linline{Function}, \linline{Function1}, \linline{Function2}, etc.
\begin{lstlisting}
trait Function1[-T1, +R] {
  def apply(p: T1): R
}
\end{lstlisting}
This allows function types be compatible to each other.
\begin{example}
\begin{lstlisting}
def process(f: ChildP => BaseR)

def foo(v: BaseP): ChildR = { ... }

// We can call process with compatible type
process(foo)
\end{lstlisting}
\end{example}

\end{frame}

\begin{frame}[fragile]
\frametitle{Collection framework}
Collection framework current implementation was introduced in Scala 2.8.  It provides two implementations:

\begin{itemize}
\item Immutable.  Default collections.  Once created it cannot be changed.
\item Mutable.  Has modification operations with side effects.
\end{itemize}

Also you can use Java collections, but Scala collections looks more useful, and provide lot of functional features.
\end{frame}

\begin{frame}[fragile]
\frametitle{Collections example}
\begin{lstlisting}
val lst = List(2, 3, 5, 26, 7, 4, 13)

val oddOnly = lst.filter(x => x % 2 == 1)

val lesTenDbl = lst.filter(_ < 10).map(_ * 2)

val sqsum = lst.foldLeft(0)((a, i) => a + i * i)

val other = List(5,12,6,42)
val result = lst.zip(other).map(v => v._1 + v._2).sorted
result.zipWithIndex.foreach(c => println(s"${c._2}. ${c._1}")

case class Person(name: String, age: Int)
val people = List(Person("Bob", 25), Person("Alice", 22))
println("Sorted by age:" + people.sortBy(_.age))
println("Longest name:" + people.maxBy(_.name.length))

val peopleByAge = people.groupBy(_.age) // Map[Int, List[Person]]

val (young, old) = people.partition(_.age < 25) // (List[Person], List[Person])

var words = Map.empty[String, Int]
words += "Word" -> 5
\end{lstlisting}
\end{frame}

\begin{frame}[fragile]
\frametitle{Streams}
Stream is a \emph{lazy} list where elements evaluated only when they are needed. \linline{#::} constructs stream
from it's head and tail.  Streams can be used to generate endless sequences.

\begin{lstlisting}
val rand = {
  def iter(): Stream[Int] = Random.nextInt #:: iter
  iter
}

rand takeWhile (_ > 100) foreach println

val evens = {
  def it(x: Int): Stream[Int] = x #:: it(x+2)
  it(0)
}

val fibs: Stream[BigInt] = 0 #:: 1 #:: fibs.zip(fibs.tail).map(x => x._1 + x._2)
\end{lstlisting}

\end{frame}

\begin{frame}[fragile]
\frametitle{Partially applied functions}

In Scala functions can have several lists of parameters. If you pass parameters not for lists of parameters
you will get a function of the rest parameters.

\begin{example}
\begin{lstlisting}
def modN(n: Int)(x: Int) = x % n == 0

// return false
modN(5)(3)

// returns List(4, 28)
List(1,2,4,15,7,28).filter(modN(4))
\end{lstlisting}
\end{example}
\end{frame}

\begin{frame}[fragile]
\frametitle{Pattern matching}

\linline{Option[T]} is a class which can hold some value of type T, or hold None. It's extremely useful for return values
(instead of return null and get NullPointerException). For example List.find method returns Option. Options can be pattern matched.

\begin{lstlisting}
  val names = List("Ann", "Roger", "Bob", "John")

  def findByPrefix(prefix: String) = {
     names.find(_.startsWith(prefix)) match {
      case Some(name) => "It's " + name
      case None => "Nobody found"
    }
  }

  println(findByPrefix("B"))

  val (beforeG, afterG) = names partition (_ <= "G")

  def quickSort[T](l: List[T])(implicit ord: Ordering[T]): List[T] = l match {
    case x :: xs => 
      val (l, h) = xs partition (v => ord.lteq(v, x))
      quickSort(l) ::: x :: quickSort(h)

    case Nil => Nil
  }
\end{lstlisting}

\end{frame}

\begin{frame}[fragile]
\frametitle{Monads}

Each monad class supports following operations: \linline{map}, \linline{flatMap}, and wrap into monadic type (constructor). For example the \linline{Option} class has:

\begin{lstlisting}
sealed abstract class Option[+A] {
  def map[B](f: A => B): Option[B] =
    if (isEmpty) None else Some(f(this.get))

  def flatMap[B](f: A => Option[B]): Option[B] =
    if (isEmpty) None else f(this.get)

  def get: A

  def getOrElse[B >: A](default: => B): B =
    if (isEmpty) default else this.get
}
|\pause|
case class Some[+A] extends Option[A] {
  def isEmpty = false
  def get = x
}
|\pause|
case class None extends Option[Nothing] {
  def isEmpty = true
  def get = throw new NoSuchElementException("None.get")
}
\end{lstlisting}

\end{frame}

\begin{frame}[fragile]
\frametitle{Use Option as monad example}

\begin{lstlisting}
object DB {
  def getPhone(name: String): Option[String]
  def getName(email: String): Option[String]
  def getEmail(username: String): Option[String]
}
\end{lstlisting}

\pause

\begin{columns}[t]
\begin{column}{0.5\textwidth}
\begin{lstlisting}
def getPhoneByUser(user: String) = {
  DB.getEmail(username) match {
    case Some(email) => 
      DB.getName(email) match {
        case Some(n) => getPhone(n)
        case None => None
    }
    case None => None
  }
}
\end{lstlisting}
\pause
\end{column}
\begin{column}{0.5\textwidth}
\begin{lstlisting}
def getPhoneByUser(user: String) =
  DB.getEmail(user)
    .flatMap(DB.getName)
    .flatMap(DB.getPhone)
\end{lstlisting}
\end{column}
\end{columns}

\end{frame}

\begin{frame}[fragile]
\frametitle{List is a monad too}
List (and other \linline{Iterable}s is also a monad. It contains method \linline{def flatMap(f: A => List[B]): List[B]}.
It applies the function and then \linline{flatten}s the list.

Here is the example of chess kinght moving:
\begin{lstlisting}
type Position = (Int, Int)

def nextPositions(current: Position): List[Position] = {
  val (x, y) = current
  val all = List((x + 2, y - 1), (x + 2, y + 1),
                 (x - 2, y - 1), (x - 2, y + 1),
                 (x - 1, y - 2), (x - 1, y + 2),
                 (x + 1, y - 2), (x + 1, y + 2))
  all.filter(p => (1 to 8).contains(p._1) && (1 to 8).contains(p._2))
}

def twoMoves(pos: (Int, Int)) = {
  val first = nextPositions(pos)
  first ++ first.flatMap(nextPositions)
}
\end{lstlisting}

\end{frame}

\begin{frame}[fragile]
\frametitle{Try. Not a monad but looks like}

\linline{Try} wraps a computation which can be failed.

\begin{lstlisting}
// Can throw IOException
def readFile(fileName: String): String

// Can throw some other network exceptions
def sendByMail(address: InternetAddress, subject: String, content: String)

def sendFile(file: String, addr: InternetAddress, subject: String): Boolean =
  Try(readFile(fileName)).flatMap(data =>
    Try(sendByMail(address, subject, data))).isSuccess

\end{lstlisting}

\end{frame}

\begin{frame}[fragile]
\frametitle{For comprehensions}
\begin{columns}[t]
\begin{column}{0.48\textwidth}
\begin{lstlisting}
(1 to 10).foreach(i => println("i = " + i))
\end{lstlisting}
\end{column}
\begin{column}{0.48\textwidth}
\begin{lstlisting}
for {
  i <- 1 to 10
} {
  println("i = " + i)
}
\end{lstlisting}
\end{column}
\end{columns}

\pause

\begin{columns}[t]
\begin{column}{0.48\textwidth}
\begin{lstlisting}
val a = List("orange", "apple", "pear")
val lengths = a.map(_.length)
\end{lstlisting}
\end{column}
\begin{column}{0.48\textwidth}
\begin{lstlisting}
val a = List("orange", "apple", "pear")
val lengths = for {
  x <- a
} yield x.length
\end{lstlisting}
\end{column}
\end{columns}

\pause

\begin{columns}[t]
\begin{column}{0.48\textwidth}
\begin{lstlisting}
def getPhoneCodeByUser(user: String) =
  DB.getEmail(user)
    .flatMap(DB.getName)
    .flatMap(DB.getPhone)
    .map(_.substring(0,3))
\end{lstlisting}
\end{column}
\begin{column}{0.48\textwidth}
\begin{lstlisting}
def getPhoneCodeByUser(user: String) =
  for {
    email <- DB.getEmail(user)
    name <- DB.getName(email)
    phone <- DB.getPhone(name)
  } yield phone.substring(0,3)
\end{lstlisting}
\end{column}
\end{columns}

\end{frame}

\section{Tools and libraries}

\begin{frame}{Scala tools and libraries}
There are lot of useful tools and libraries for Scala:
\begin{itemize}
\item \href{http://www.scala-lang.org}{Scala home} -- Scala language home site
\item \href{http://www.scala-sbt.org}{sbt} -- build tool for Scala, Java and more
\item \href{http://www.akka.io}{Akka} -- actor toolkit
\item \href{http://spark.apache.org}{Spark} -- big data analytics engine
\item \href{http://www.playframework.com}{Play Framework} -- web framework
\item \href{http://www.liftweb.net}{Lift} -- web framework
\item \href{http://slick.typesafe.com}{Slick} -- database query and access library
\item \href{http://www.scalastyle.org/}{ScalaStyle} -- static code analyzer
\item \href{http://www.scalatest.org}{ScalaTest} -- unit test framework
\item \href{http://www.scalacheck.org}{ScalaCheck} -- property-based testing library
\item \href{http://www.scalamock.org}{ScalaMock} -- mocking library
\end{itemize}
\end{frame}

\begin{frame}{The end}
\centering
\textbf{That's all folks!}

Code examples and this presentation are available at \url{http://github.com/limansky/trainings}\\~\\

\scriptsize{\ccbysa \ Mike Limanskiy, 2014-2015.}
\vfill
\tiny
This work is licensed under a \href{http://creativecommons.org/licenses/by-sa/4.0/}{Creative Commons Attribution-ShareAlike 4.0 International License}.

Created with \LaTeXe.
\end{frame}



\end{document}
