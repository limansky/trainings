\documentclass{beamer}
\usepackage[utf8]{inputenc}

\documentclass[t]{beamer}

\usepackage[utf8]{inputenc}

\usepackage{color}
\definecolor{sh_comment}{rgb}{0.12, 0.38, 0.18 } %adjusted, in Eclipse: {0.25, 0.42, 0.30 } = #3F6A4D
\definecolor{sh_keyword}{rgb}{0.37, 0.08, 0.25}  % #5F1441
\definecolor{sh_string}{rgb}{0.06, 0.10, 0.98} % #101AF9
\definecolor{links}{HTML}{2A1BFF}

\usepackage[T1]{fontenc}
\usepackage[lighttt]{lmodern}

\usepackage{hyperref}
\usepackage{listings}
\usepackage{ccicons}

% "define" Scala
\lstdefinelanguage{Scala}{
  morekeywords={abstract,case,catch,class,def,%
    do,else,extends,false,final,finally,%
    for,if,implicit,import,match,mixin,%
    new,null,object,override,package,%
    private,protected,requires,return,sealed,%
    super,this,throw,trait,true,try,%
    type,val,var,while,with,yield},
  otherkeywords={=>,<-,<\%,<:,>:,\#,@},
  sensitive=true,
  morecomment=[l]{//},
  morecomment=[n]{/*}{*/},
  morestring=[b]",
  morestring=[b]',
  morestring=[b]"""
}

\lstset{
  language=Scala,
  rulesepcolor=\color{black},
  rulesepcolor=\color{black},
  showspaces=false,showtabs=false,tabsize=2,showstringspaces=false,
  numberstyle=\tiny,numbers=left,
  basicstyle=\ttfamily\tiny,
  stringstyle=\color{sh_string}, 
  keywordstyle=\color{sh_keyword}\bfseries,
  commentstyle=\color{sh_comment}, %\itshape
  captionpos=b,
  xleftmargin=0.7cm, xrightmargin=0.5cm,
  lineskip=-0.3em,
  escapebegin={\lstsmallmath}, escapeend={\lstsmallmathend}
}

\hypersetup{colorlinks,linkcolor=,urlcolor=links}

\def\linline{\lstinline[basicstyle=\ttfamily]}

\newcommand\strong\textbf

\setlength{\parskip}{\smallskipamount}

\author{Mikhail Limanskiy}

\institute{\url{www.limansky.me}}

\date{\include{date}}


\title{Futures, promises and some fun}
\subtitle{in Scala and C++}

\author{Mikhail Limanskiy}

\institute{SymphonyTeleca}

\date{\today}

\begin{document}

\begin{frame}
    \titlepage
\end{frame}

\begin{frame}{Why futures?}
Futures is a modern way for parallel computations.
\begin{itemize}
\item Futures easier to use than manual thread managing.
\item Code is easier to understand, since futures wraps values.
\item Futures are available in most of the modern languages (C++11, Java 7, Scala 2.10, Python 3.2).
\end{itemize}
\end{frame}

\begin{frame}
\frametitle{Futures}
Future is an object holding a value which may become available later.  Future can have one of three states:
\begin{itemize}
\item Not yet completed
\item Completed successfully
\item Computation was failed
\end{itemize}
\end{frame}

\begin{frame}[fragile]
\frametitle{Future class}
\linline{Future} instances can be created using \linline{future} construct. 
To use futures you need to have an instance of \linline{ExecutionContext}. You can just import global one.
Future has several callbacks:
\begin{lstlisting}
trait Future[+T] extends Awaitable[T] {
  onComplete[U](func: (Try[T]) => U): Unit
  onSuccess[U](pf: PartialFunction[T, U]): Unit
  onFailure[U](pf: PartialFunction[Throwable, U]): Unit
}
\end{lstlisting}
\end{frame}

\begin{frame}[fragile]
\frametitle{Example}
\begin{lstlisting}
import scala.io.Source
import scala.concurrent._
import scala.concurrent.ExecutionContext.Implicits.global
import scala.util.{Success, Failure}

object Fetcher extends App {
  def getPage(url: String): Future[String] = {
    future {
      Source.fromURL(url).mkString
    }
  }

  args.foreach { url =>
    val f = getPage(url)

    f onComplete {
      case Success(p) => println(url + ": " + p.take(200))
      case Failure(_) => println("Failed to get content from " + url)
    }
  }
}
\end{lstlisting}
\end{frame}

\begin{frame}[fragile]
\frametitle{Fetcher example in C++ 11}
\begin{lstlisting}[language=C++]
#include <future>
#include <iostream>
#include <vector>
#include <curl/curl.h>

using namespace std;
using CurlWriter = size_t(*)(char*, size_t, size_t, string*);

string getPage(const char* url) {
    CURL* curl = curl_easy_init();
    string result;
    curl_easy_setopt(curl, CURLOPT_URL, url);
    curl_easy_setopt(curl, CURLOPT_FOLLOWLOCATION, 1);
    curl_easy_setopt(curl, CURLOPT_WRITEFUNCTION, static_cast<CurlWriter>(
        [](char* buf, size_t s, size_t n, string* r) {
            r->append(buf);
            return s * n;
        }));
    curl_easy_setopt(curl, CURLOPT_WRITEDATA, &result);
    curl_easy_perform(curl);
    curl_easy_cleanup(curl);
    return result;
}
int main(int argc, char* argv[]) {
    vector<pair<const char*, future<string>>> results;
    for (int i = 0; i < argc; ++i)
        results.push_back(make_pair(argv[i], async(getPage, argv[i])));
    for(auto& r : results)
        cout << r.first << ": " << r.second.get().substr(0, 200) << "\n";
    return 0;
}
\end{lstlisting}
\end{frame}

\begin{frame}[fragile]
\frametitle{Futures composition}
\linline{Future} is a monad. You can use \linline{map}, \linline{flatMap}, \linline{filter}, etc to
compose \linline{Future}s together. You can also use for-comprehensions.
\end{frame}

\begin{frame}
\frametitle{Promises}
\end{frame}

\end{document}
