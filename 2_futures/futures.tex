\documentclass[t]{beamer}

\usepackage[utf8]{inputenc}

\usepackage{color}
\definecolor{sh_comment}{rgb}{0.12, 0.38, 0.18 } %adjusted, in Eclipse: {0.25, 0.42, 0.30 } = #3F6A4D
\definecolor{sh_keyword}{rgb}{0.37, 0.08, 0.25}  % #5F1441
\definecolor{sh_string}{rgb}{0.06, 0.10, 0.98} % #101AF9
\definecolor{links}{HTML}{2A1BFF}

\usepackage[T1]{fontenc}
\usepackage[lighttt]{lmodern}

\usepackage{hyperref}
\usepackage{listings}
\usepackage{ccicons}

% "define" Scala
\lstdefinelanguage{Scala}{
  morekeywords={abstract,case,catch,class,def,%
    do,else,extends,false,final,finally,%
    for,if,implicit,import,match,mixin,%
    new,null,object,override,package,%
    private,protected,requires,return,sealed,%
    super,this,throw,trait,true,try,%
    type,val,var,while,with,yield},
  otherkeywords={=>,<-,<\%,<:,>:,\#,@},
  sensitive=true,
  morecomment=[l]{//},
  morecomment=[n]{/*}{*/},
  morestring=[b]",
  morestring=[b]',
  morestring=[b]"""
}

\lstset{
  language=Scala,
  rulesepcolor=\color{black},
  rulesepcolor=\color{black},
  showspaces=false,showtabs=false,tabsize=2,showstringspaces=false,
  numberstyle=\tiny,numbers=left,
  basicstyle=\ttfamily\tiny,
  stringstyle=\color{sh_string}, 
  keywordstyle=\color{sh_keyword}\bfseries,
  commentstyle=\color{sh_comment}, %\itshape
  captionpos=b,
  xleftmargin=0.7cm, xrightmargin=0.5cm,
  lineskip=-0.3em,
  escapebegin={\lstsmallmath}, escapeend={\lstsmallmathend}
}

\hypersetup{colorlinks,linkcolor=,urlcolor=links}

\def\linline{\lstinline[basicstyle=\ttfamily]}

\newcommand\strong\textbf

\setlength{\parskip}{\smallskipamount}

\author{Mikhail Limanskiy}

\institute{\url{www.limansky.me}}

\date{\include{date}}


\usepackage{tikz}
\usepackage{tikz-uml}

\title{Futures, promises and some fun}
\subtitle{in Scala and C++}

\begin{document}

\begin{frame}
    \titlepage
\end{frame}

\section {Intro}

\begin{frame}{Why futures?}
Futures is a modern way for parallel computations.
\begin{itemize}
\item Futures easier to use than manual thread managing.
\item Code is easier to understand, since futures wraps values.
\item Futures are available in most of the modern languages (C++11, Java 7, Scala 2.10, Python 3.2).
\end{itemize}
\end{frame}

\section{Futures}

\begin{frame}
\frametitle{Futures}
Future is an object holding a value which may become available later.  Future can have one of three states:
\begin{itemize}
\item Not yet completed
\item Completed successfully
\item Computation was failed
\end{itemize}
\end{frame}

\begin{frame}[fragile]
\frametitle{Future class}
\linline{Future} instances can be created using \linline{Future} construct.
To use futures you need to have an instance of \linline{ExecutionContext}. You can just import global one.
Future has several callbacks:
\begin{lstlisting}
trait Future[+T] extends Awaitable[T] {
  onComplete[U](func: (Try[T]) => U): Unit
  onSuccess[U](pf: PartialFunction[T, U]): Unit
  onFailure[U](pf: PartialFunction[Throwable, U]): Unit
}
\end{lstlisting}
\end{frame}

\begin{frame}[fragile]
\frametitle{Waiting for the result}
Sometimes you need to block current thread and wait for the future result.
It can be achieved using \linline{Await.result} or \linline{Await.ready} method:
\begin{example}
\begin{lstlisting}
import scala.concurrent._
import scala.concurrent.duration._

def getWeather(date: Date): Future[WeatherReport]

val weatherFuture = getWeather(new Date)

// Doing something else

val report = Await.result(weatherFuture, 10.seconds)
\end{lstlisting}
\end{example}
\end{frame}

\begin{frame}[fragile]
\frametitle{Example}
\begin{lstlisting}
import scala.io.Source
import scala.concurrent._
import scala.concurrent.duration.Duration
import scala.concurrent.ExecutionContext.Implicits.global
import scala.util.{Success, Failure}

object Fetcher extends App {
  def getPage(url: String): Future[String] = {
    Future {
      Source.fromURL(url).mkString
    }
  }

  val tasks = Future.sequence(args.map { url =>
    val f = getPage(url)

    f onComplete {
      case Success(p) => println(url + ": " + p.take(200))
      case Failure(_) => println("Failed to get content from " + url)
    }

    f
  })

  Await.ready(tasks, Duration.Inf)
}
\end{lstlisting}

Here I'm using \linline{Future.sequence} to transform \linline{List[Future[String]]} to
\linline{Future[List[String]]}.
\end{frame}

\begin{frame}[fragile]
\frametitle{C++ std::future}
In C++11 \linline{std::future} class has following methods:
\begin{itemize}
\item \linline{get} -- provides value if it ready or blocks current thread.
\item \linline{wait} -- blocks current thread while value is not ready.
\item \linline{wait_for}, \linline{wait_until} -- wait for result, but not longer than
required.
\end{itemize}
\linline{std::async} function can be used to create future from function, method or function object.
\end{frame}

\begin{frame}[fragile]
\frametitle{Fetcher example in C++ 11}
\begin{lstlisting}[language=C++]
#include <future>
#include <iostream>
#include <vector>
#include <curl/curl.h>

using namespace std;
using CurlWriter = size_t(*)(char*, size_t, size_t, string*);

string getPage(const char* url) {
    CURL* curl = curl_easy_init();
    string result;
    curl_easy_setopt(curl, CURLOPT_URL, url);
    curl_easy_setopt(curl, CURLOPT_FOLLOWLOCATION, 1);
    curl_easy_setopt(curl, CURLOPT_WRITEFUNCTION, static_cast<CurlWriter>(
        [](char* buf, size_t s, size_t n, string* r) {
            r->append(buf);
            return s * n;
        }));
    curl_easy_setopt(curl, CURLOPT_WRITEDATA, &result);
    curl_easy_perform(curl);
    curl_easy_cleanup(curl);
    return result;
}
int main(int argc, char* argv[]) {
    vector<pair<const char*, future<string>>> results;
    for (int i = 0; i < argc; ++i)
        results.push_back(make_pair(argv[i], async(getPage, argv[i])));
    for(auto& r : results)
        cout << r.first << ": " << r.second.get().substr(0, 200) << "\n";
    return 0;
}
\end{lstlisting}
\end{frame}

\section{Promises}

\begin{frame}
\frametitle{Promises}
\linline{Promise} is a kind of container, which owns a \linline{Future} object and completes it
when the value is ready.
\begin{enumerate}
\item Create future of proper data type.
\item Start a computation.
\item Return future to the caller using \linline{Promise.future}.
\item Complete promise using \linline{success} or \linline{failure} method.
\end{enumerate}
\end{frame}

\begin{frame}[fragile]
\frametitle{Promises life cycle}
\begin{tikzpicture}
\begin{umlseqdiag}
\umlactor[scale=0.5]{actor}
\umlobject[class=Foo, x=3]{foo}
\begin{umlcall}[op={getData()},return={f}]{actor}{foo}
  \umlcreatecall[class=Promise,x=6]{foo}{p}
  \umlcreatecall[class=Future, x=9]{p}{f}
\end{umlcall}
\begin{umlcallself}[op=calc()]{foo}
\begin{umlcall}[type=asynchron,op=success(v)]{foo}{p}
\begin{umlcall}[op=complete(v)]{p}{f}
\end{umlcall}
\end{umlcall}
\end{umlcallself}
\end{umlseqdiag}
\end{tikzpicture}
\end{frame}

\begin{frame}[fragile]
\frametitle{Code example}
For example we need to get result of the first completed future from list:
\begin{lstlisting}
def firstResultOf[T](futures: List[Future[T]]): Future[T] = {
  val p = Promise[T]()
  futures.foreach(_ onComplete { p.tryComplete(_) })
  p.future
}
\end{lstlisting}
If we are intrested only in success result:
\begin{lstlisting}
def firstSuccessOf[T](futures: Future[T]...): Future[T] = {
  val p = Promise[T]()
  left = futures.size
  futures.foreach(_ on Complete {
    case Success(v) => p.trySuccess(v)
    case Failure(e) =>
      if (left == 0) p.tryFailure(e)
      left -= 1
  }
  p.future
}
\end{lstlisting}
\end{frame}

\begin{frame}[fragile]
\frametitle{C++ std::promise}
In C++11 \linline{std::promise} class provides a way to pass values between threads.
Most important methods are:
\begin{itemize}
\item \linline{get_future} -- returns future object.
\item \linline{set_value} -- completes future with value.
\item \linline{set_exception} -- completes future with exception.
\end{itemize}
\end{frame}

\begin{frame}[fragile]
\frametitle{C++ example}
For example we want to search char sequence in file asynchronously:
\begin{lstlisting}[name=bingrep,language=C++]
using namespace std;

static const size_t BUFFER_SIZE = 1024 * 1024;

using pattern = vector<char>;

void scanfile(const char* filename, const pattern& pat, promise<int>& p) {
    ifstream f(filename, ios::binary);
    int begin = 0;
    vector<char> buf(BUFFER_SIZE);
    try {
        while (f) {
            f.read(buf.data(), BUFFER_SIZE);
            auto end = buf.begin() + f.gcount();
            auto pos = search(buf.begin(), end, pat.begin(), pat.end());
            if (pos != end) {
                p.set_value(begin + (pos - buf.begin()));
                return;
            }
            if (f) {
                begin += f.gcount() - pat.size();
                f.seekg(-pat.size(), ios_base::cur);
            }
        }
    } catch (exception&) {
        p.set_exception(current_exception());
    }
    p.set_value(-1);
}
\end{lstlisting}
\end{frame}

\begin{frame}[fragile]
\frametitle{C++ example}
\begin{lstlisting}[name=bingrep,language=C++]
int main(int argc, char* argv[]) {
    if (argc < 2) {
        cout << "Usage: bingrep <filename> <hex bytes*>\n";
        return -1;
    }

    pattern pat;
    for (int i = 2; i < argc; ++i) {
        char* byte = argv[i];
        if (strlen(byte) != 2) {
            cout << "Invalid byte: " << byte << "\n";
            return -3;
        }
        pat.push_back(stoul(byte, nullptr, 16));
    }

    promise<int> p;
    thread worker(&scanfile, argv[1], pat, ref(p));

    auto result = p.get_future();
    chrono::milliseconds span(100);
    cout << "Searching: ";
    while (result.wait_for(span) == future_status::timeout) cout << '.' << flush;

    int pos = result.get();
    if (pos == -1)
        cout << "\nPattern not found\n";
    else
        cout << "\nPattern found at " << pos << "\n";

    worker.join();
    return 0;
}
\end{lstlisting}
\end{frame}

\section{Futures composition}

\begin{frame}[fragile]
\frametitle{For-comprehensions}
In scala \linline{for} is an alternative way to call: \linline{map}, \linline{flatMap},
\linline{foreach}, \linline{filter}.

In many cases this syntax is more readable:
\lstset{numbers=none}
\begin{table}
\renewcommand{\arraystretch}{1.5}
\begin{tabular*}{0.9\textwidth}{ l l }
\hline
For comprehension & Functional \\ \hline

\begin{lstlisting}
for (i <- 1 to 5) {
  println(i)
}
\end{lstlisting}
&
\begin{lstlisting}
1 to 5 flatmap (i => println(i))
\end{lstlisting}
\\ \hline
\begin{lstlisting}
val a = for {
  i <- list
} yield i.toString
\end{lstlisting}
&
\begin{lstlisting}
val a = list.map(_.toString)
\end{lstlisting}
\\ \hline
\begin{lstlisting}
val a = for {
  i <- list1
  j <- list2
} yield i*j
\end{lstlisting}
&
\begin{lstlisting}
val a = list1.flatMap(i =>
  list2.map(_ * i))
\end{lstlisting}
\\ \hline
\begin{lstlisting}
val a = for {
  i <- list if (i % 2 == 0)
} yield i / 2
\end{lstlisting}
&
\begin{lstlisting}
val a = list.filter(_ %2 == 0).map(_ / 2)
\end{lstlisting}
\end{tabular*}
\end{table}

This syntax can be used not only with List, but also with other sequences,
Options, and other any classes, which provides required methods.
\end{frame}

\begin{frame}[fragile]
\frametitle{Futures composition}
\linline{Future} is a monad. You can use \linline{map}, \linline{flatMap}, \linline{filter}, etc to
compose \linline{Future}s together. Since these functions are aveailable you can also use for-comprehensions.
\begin{lstlisting}
def getPageSize(url: String): Future[Int] = getPage(url).map(_.length)

val linkedPage = getPage(url).flatMap { page =>
  val firstUrl = findUrl(page)
  if (firstUrl.nonEmpty) getPage(firstUrl)
  else throw new Exception("Page with no links")
}
\end{lstlisting}
\end{frame}

\begin{frame}[fragile]
\frametitle{More combinators}
There are some additional combinators, helps you handle failures:
\begin{itemize}
\item \linline{fallbackTo[U >: T](that: Future[U])} -- returns that if this failed.
\item \linline{recover[U >: T](f: Throwable => U)} -- completes failed future with value returned by f.
\item \linline{recoverWith[U >: T](f: Throwable => Future[U])} -- returns value returned by f if original future failed.
\end{itemize}
\begin{example}
\begin{lstlisting}
val countRecords: Future[Int] = readRecordsFromFile(filename) map (
    _.size
  ) recover {
    case FileNotFoundException => 0
  }
\end{lstlisting}
\end{example}
\end{frame}

\begin{frame}[fragile]
\frametitle{Multipart fetcher example}
We will use Dispatch library, which provides asynchronous HTTP API (see \cite{dispatch}).
\begin{lstlisting}[name=multifetch]
val partsNumber = 5

def getSinglePart(link: String): Future[Array[Byte]] = {
  Http(url(link) OK as.Bytes)
}

def getMultipart(link: String): Future[Array[Byte]] = {

  def getPart(from: Int, to: Option[Int]): Future[Array[Byte]] = {
    val toStr = to.map(_.toString).getOrElse("")
    val u = url(link) <:< Map("range" -> s"bytes=$from-$toStr")
    Http(u OK as.Bytes)
  }

  val f = for {
    resp <- Http(url(link).HEAD)
    h = resp.getHeaders()
    if h.containsKey("Accept-Ranges") && h.containsKey("Content-Length")
  } yield h.getFirstValue("Content-Length").toInt

  f flatMap { size =>
    val partSize = size / partsNumber
    val tasks = (0 to partsNumber - 2).map(i =>
                  getPart(i * partSize, Some((i + 1) * partSize - 1))
                ) :+ getPart((partsNumber - 1) * partSize, None)
    Future.sequence(tasks)
  } map {
    _.toArray.flatten
  }
}
\end{lstlisting}
\end{frame}

\begin{frame}[fragile]
\frametitle{Multipart fetcher, part 2}
Now it's possible to fetch a file using multipart fetcher, but if it fails make attempt to use
regular GET requiest.
\begin{lstlisting}[name=multifetch]
args.toList match {
  case link :: Nil =>
    val f = getMultipart(link) fallbackTo {
      getSinglePart(link)
    } 
    
    f onComplete {
      case Success(data) =>
        val filename = link.substring(link.lastIndexOf('/') + 1)
        val fs = new FileOutputStream(filename)
        fs.write(data)
        fs.close
        println("Done")
      case Failure(t) =>
        println(s"Failed to fetch file from $link, because of ${t.getMessage}")
    }
    Await.ready(f, Duration.Inf)
  case _ => println("Usage: multiget <link>")
}
\end{lstlisting}
\end{frame}


\begin{frame}{Bibliography}
\begin{thebibliography}{00}
\bibitem{scala}Martin Odersky, Lex Spoon, Bill Venners:
\emph{Programming in Scala},
Artima, 2nd edition, 2011
\bibitem{futures}Philipp Haller, Aleksandar Prokopec, Heather Miller, Viktor Klang, Roland Kuhn, and Vojin Jovanovic:
\emph{Futures and Promises}, \url{http://docs.scala-lang.org/overviews/core/futures.html}
\bibitem{dispatch}Dispatch library, \url{http://dispatch.databinder.net/}
\end{thebibliography}
\end{frame}

\begin{frame}{The end}
\centering
\textbf{That's all folks!}

Code examples and this presentation are available at \url{http://github.com/limansky/trainings}\\~\\

\scriptsize{\ccbysa \ Mike Limanskiy, 2014-2015.}
\vfill
\tiny
This work is licensed under a \href{http://creativecommons.org/licenses/by-sa/4.0/}{Creative Commons Attribution-ShareAlike 4.0 International License}.

Created with \LaTeXe.
\end{frame}



\end{document}
